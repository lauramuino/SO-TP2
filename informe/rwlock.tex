\section{Implementación RWLock}
Se pidió realizar la implementación de la clase RWLock, que provee funciones lectura y escritura bloqueantes. 
Para hacerlo, se utilizaron Varibles de Condición POSIX. para poder garantizar con ellas el acceso exclusivo y la no inanición.
Se pasará a describir brevemente cada función.
\begin{itemize}
\item{\textbf{RWLock: }Se encarga de inicializar las variables globales que van a utilizar todas la funciones de la clase.}

\lstset{language=C,
	basicstyle=\ttfamily,
	keywordstyle=\color{blue}\ttfamily,
	stringstyle=\color{red}\ttfamily,
	commentstyle=\color{red}\ttfamily
}
\begin{lstlisting}

cant_lectores = 0;
hay_escribiendo = false;
hay_escritor_esperando = false;
    
pthread_mutex_init(&mtx_RWL, NULL);
pthread_cond_init(&barrera_lectores, NULL);		
pthread_cond_init(&primera_barrera_lectores, NULL);		
pthread_cond_init(&sem_escritores, NULL);
	
\end{lstlisting}

\item{\textbf{rlock: }}Cuando un thread llame a la función $rlock$, lo primero que va a hacer es 
\begin{lstlisting}
	
pthread_mutex_lock(&mtx_RWL);
     
while(hay_escritor_esperando){
	pthread_cond_wait(&primera_barrera_lectores, &mtx_RWL);
}
\end{lstlisting}
Toma el mutex, y se queda esperando en la $primera\_barrera\_lectores$, si es que hay algún escritor esperando. Esto es para que se respete le orden 
de llegada de los pedidos de lectura y escritura. Una vez que el thread que estaba esperando para escribir logra obtener el lock, manda un signal a todos
los lectores que estaba en la $primera\_barrera\_lectores$.
Luego los lectores deben chequear si hay alguien escribiendo en ese momento.
\begin{lstlisting}
while(hay_escribiendo){
	pthread_cond_wait(&barrera_lectores, &mtx_RWL);
}	
\end{lstlisting}
Si habia alguno el thread hace wait, sino no entra en el $while$, y aumenta en uno la $cant\_lectores$. Finalmente desbloquea el mutex.

\item{\textbf{wlock: }}Cuando un thread quiere escribir, va a llamar a esta función. 
\begin{lstlisting}
pthread_mutex_lock(&mtx_RWL);

while(cant_lectores > 0 || hay_escribiendo){
	hay_escritor_esperando = true;	//YO PASO A SER UN ESCRITOR ESPERANDO
	pthread_cond_wait(&sem_escritores, &mtx_RWL);
}
\end{lstlisting}
Lo que hace es esperar a que no haya nadie leyendo ni escribiendo para poder tener el acceso exclusivo.
Una vez que logra obetenerlo, actualiza las variables, suelta el mutex y hace broadcast a todos los lectores que estaba esperando en la
$primera\_barrera\_lectores$.
\begin{lstlisting}
hay_escritor_esperando = false;
hay_escribiendo = true;	

pthread_mutex_unlock(&mtx_RWL);
pthread_cond_broadcast(&primera_barrera_lectores);
\end{lstlisting}

\item{\textbf{runlock: }}Resta uno en la cant\_lectores y si es igual a 0, hace un signal al escritor que este esperando en \&sem\_escritores. 

\item{\textbf{wunlock: }}Pone $hay\_escribiendo = false$ y hace broadcast a todos los lectores esperando en $barrera\_lectores$ y hace signal a \&sem\_escritores.
\end{itemize}
Cada vez que se modifica una variable global, se pide el mutex.

