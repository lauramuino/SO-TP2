\section{Sincronización}
Para poder permitir la concurrencia entre jugadores, se utilizó la clase RWLock, que permite hacer lecturas simultáneas y escrituras exclusivas.
La idea de crear una matriz de estas estructuras, es protejer cada casilla por separado, permitiendo asi, modificar o leer, mas de una a la vez por distintos
threads.
\\
Entonces, cada vez que potencialmente se vaya a modificar algun casillero de algun tablero, se llama a la funcion $wlock$, que bloquea ese casillero,
sin permitir que ningun otro thread pueda ni siquiera leerlo. La funcion $rlock$ si va a permitir lecturas concurrentes.
\\
\\
En la función $atendedor\_de\_jugador$, se recibe la ficha, y la poscicion donde se quiere ubicar a la misma. A partir de ahí comienza a chequear,
si la posicón es válida, si está libre, etc.
Antes de realizar la consulta $es\_ficha\_valida\_en\_palabra$ se bloquea dicho casillero con $wlock$, ya que de ser válida, se podrá escribir en ese lugar, y
recién ahí se podrá desbloquear para garantizar el acceso exclusivo a esa pos.
\\
\\
\\


Dentro de la función $es\_ficha\_valida\_en\_palabra$, se chequea que la poscición en la que se desa ubicar la letra sea válida, es decir, que esté dentro
del tablero, que no está ocupada, que si no es la primera letra de la $palabra\_actual$ sea solo vertical o solo horizontal, y que, si hay espacio entre
la letra anterior y la que estoy poniendo, ese espacio esta ocupado por otras letras en el $tablero\_palabras$. Es decir, esta usando letras de otra palabra ya
escrita en el tablero. 
En cualquier caso que no cumpla las condiciones, se debe desbloquear esa posicion del $tablero\_letras$, osea hacer $wunlock$ en la matriz $lock\_casilla\_temp[f][c]$.
Si se cumplieran todas la condiciones, entonces la letra puede escribirse en esa pos, por lo que se libera el lock, despues de modificarla.
Cuando se realizan los chequeos de los casilleros del $tablero\_palabras$, se realizan $rlock$, ya que como no se vana  modficar, esto permita que varios threads
puedan leer ese valor. Una vez que se termina de acceder, se hace $runlock$, para desbloquearlo.
Básicamente se bloquean las casillas cada vez se quiere hacer una escritura o una lectura.
En la funcione $enviar\_tablero$, se hace $rlock$ sobre lock\_casilla\_temp ya que varios jugadores pueden estar actualizando el tablero a la vez.
En $quitar\_letras$, se modifica el $tablero\_letras$ entoces se hace $wlock$ sobre lock\_casilla\_temp y en $atendedor\_de\_jugador$ cuando se recibe el commando MSG\_PALABRA, copia la $palabra\_actual$ en el 
$tablero\_palabras$, para esto hacemos $wrlock$ sobre lock\_casilla\_posta. Luego de hacer la lectura o escritura, de libera la casilla haciendo el unlock correspondiente.


