\section{Servidor Multi}

Para crear un servidor que soporte mútiples jugadores, se adaptó el backend-mono provisto por la cátedra.
El primer cambio es permitir que atienda mas de una conexión. Para esto se utilizan $threads$. La idea es que cada vez que un cliente se conecte al servidor,
se cree un thread nuevo que va a atenderlo hasta que termine de jugar, y luego de eso terminará exitosamente. 
Otro problema a solucionar es la sincronización entre threads, ya que ahora van a estar compartiendo el tablero, y "compitiendo" por cada casilla para
poder leer y escribir.Por otro lado, se desea no perder demasiada concurrencia en el juego, es decir, que deberian poder modificar simultáneamente dos posiciones
distintas de las tableros simultáneamente, sin tener problemas.  

A continuación se muestran brevemente los cambios realizados.
Las variables globales utilizadas son:
\lstset{language=C,
	basicstyle=\ttfamily,
	keywordstyle=\color{blue}\ttfamily,
	stringstyle=\color{red}\ttfamily,
	commentstyle=\color{red}\ttfamily%,
	%morecomment=[1][\color{magenta}]{\#}
}
\begin{lstlisting}

int socket_servidor = -1;

//matriz de locks para proteger cada casilla del tablero_palabras
vector<vector<RWLock> > lock_casilla_posta;	

//matriz de locks para proteger cada casilla del tablero_letras
vector<vector<RWLock> > lock_casilla_temp;	

// tiene letras que aun no son palabras validas
vector<vector<char> > tablero_letras; 		
// solamente tiene las palabras validas
vector<vector<char> > tablero_palabras;		

unsigned int ancho = -1;
unsigned int alto = -1;
\end{lstlisting}
Estas se inicializan en la función main, una vez que se obtienen los valores de alto, y ancho del tablero. 


Aqui un fragmento de código de backend-multi.cpp:

\lstset{language=C,
	basicstyle=\ttfamily,
	keywordstyle=\color{blue}\ttfamily,
	stringstyle=\color{red}\ttfamily,
	commentstyle=\color{red}\ttfamily%,
	%morecomment=[1][\color{magenta}]{\#}
}
\begin{lstlisting}

while (true) {
    if ((socketfd_cliente = accept(socket_servidor, (struct sockaddr*) &remoto,
					(socklen_t*) &socket_size)) == -1)
        cerr << "Error al aceptar conexion" << endl;
    else {
		int* nuevo_socket = new int;
		*nuevo_socket = socketfd_cliente;
		pthread_t thread_id;	
			
//Aca se crea un nuevo thread que va a atender la conexion nueva(el jugador nuevo)
		pthread_create(&thread_id, NULL, atendedor_de_jugador,
					(void*) nuevo_socket);
    }
}
\end{lstlisting}
\\

La función $pthread\_create$ se encarga de crear un thread nuevo, el cual va empezar su ejecución en la función $atendedor\_de\_jugador$, y se le pasa el argumento
$nuevo\_socket$, que contiene el file descriptor del socket del nuevo cliente.

El thread principal, que es el servidor, se va a quedar escuchando conexiones entrantes de nuevos jugadores y se va a encargar de crear los threads. 
\\
Una vez creado el thread, comienza a ejecutar en la función $atendedor\_de\_jugador$. Lo primero que ejecuta es la función $pthread\_detach(pthread\_self())$, 
Esto hace que cuando el thread termine y haga exit, devuelva todos los recursos ultilizados, sin necesidad de que el thread principal haga join con el que
esta terminando.
Luego se obtiene el numero de socket del cliente que esta atendiendo ese thread para poder realizar el envío de mensajes con el servidor.
Finalmente, cuando el cliente elija terminar el juego, se llamará a la función $terminar\_servidor\_de\_jugador$, la cual cierra el socket del cliente terminado,
libera los casilleros del $tablero\_letras$ que pudiera estar utilizando, y llama a la función $pthread\_exit$, que termina el thread de manera exitosa.
\\ 
\\
\\
NOTA: cuando el servidor termina, tambien cierra su socket y hace exit. Sería mas adecuado que cierre los sockets de todos los clientes que estan corriendo
y ver que terminen exitosamente. Esto no afecta el funcionamiento del juego.


